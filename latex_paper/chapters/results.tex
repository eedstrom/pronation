Data were collected from three varying tests from two subjects.
Displayed below are data of the angular displacement and force graphs taken for Subject 1 in a neutral standing stance.
The initial values differ because the filter is calibrating the data.\par

\begin{figure}[p]
  \centering
  \includegraphics[scale=0.3]{still_data}
  \caption[Stationary stance data]{Stationary stance data}
  \label{fig:x neutral data}
\end{figure}

The vertical violet lines represent when the subject’s heel strikes the ground.
In other words, when a new step is taken.
However, because these are stationary data, this line represents a slight shift of the subject’s body weight.
That is why the plot’s values vary on either side of that line.\par

\begin{figure}[h]
  \centering
  \includegraphics[scale=0.18]{neutral_stance_frame}
  \caption[Stationary stance frame]{Video frame of stationary standing stance}
  \label{fig:x neutral frame}
\end{figure}

The next data were of the subjects walking at a slow, controlled pace.\par

\begin{figure}[p]
  \centering
  \includegraphics[scale=0.3]{walk_color}
  \caption[Slow walking data]{Slow controlled walking data}
  \label{fig:x slow data}
\end{figure}

\begin{figure}[h]
  \centering
  \includegraphics[scale=0.18]{slow_walking_frame}
  \caption[Slow walking frame]{Video frame of slow controlled walking}
  \label{fig:x slow frame}
\end{figure}

The steps were isolated using the FSR data to solidify patterns in the data. A step begins when the heel (0) hits the ground.\par

\begin{figure}[h]
  \centering
  \includegraphics[scale=0.25]{first_step}
  \caption[FSR step values]{FSR values for an individual step in the walk}
  \label{fig:x fsr step}
\end{figure}

The data collection rate for the FSRs is limited by the data collection rate of the LSM9DS1 accelerometers.
Unfortunately, the rate at which they were taking data was slower than for other datasets; however, it is fast enough to be accurate in distinguishing between steps.
\hyperref[{fig:x fsr_placement}]{Figure \ref*{fig:x fsr_placement}} maps the numeric labels of each FSR to its location in the subject’s shoe.\par

The third trial was of each of the subjects jogging at light pace on a flat, indoor surface.\par

\begin{figure}[p]
  \centering
  \includegraphics[scale=0.3]{brian_jog}
  \caption[Brisk jogging data]{Brisk jogging data}
  \label{fig:x brisk data}
\end{figure}

\begin{figure}[h]
  \centering
  \includegraphics[scale=0.18]{brisk_jogging_frame}
  \caption[Brisk jogging frame]{Video frame of brisk jogging contact point}
  \label{fig:x brisk frame}
\end{figure}
