Using a combination of force data collected using force sensitive resistors (FSRs) and angular displacement data collected using accelerometers, we were able to suggest how a person's foot might pronate or supinate during dynamic motion.
None of our subjects showed any strongs signs of pronation from the data.
Even in the cases where we had the subject purposely pronate and supinate, the effect was minimal.
While the sample size of this experiment is small, the creation of a low-cost device to measure pronation and supination could prove useful in further studies with larger sample sizes, especially if subjects who have a known medical history of pronation or supination could be gathered.
