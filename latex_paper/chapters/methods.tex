\section{Overview}
We used an Arduino Mega 2560 along with three Adafruit LSM9DS1 units and four force-sensitive resistors to create our data logger for the data collection.
The LSM9DS1s each contain an accelerometer, a gyroscope, and a magnetometer, which were used to measure the AOP and AOS.
The force-sensitive resistors enabled us to measure the force from regions of the subject’s foot.
These instruments communicated using the I2C protocol and were physically attached to the subject’s leg using elastic bandages.
We then analyzed the data using Python code to determine a subject’s level of pronation or supination.
Using this method, we were able to accurately and repeatably measure the strides of various subjects.
An explanation of these various components follows.\par

\subsection{List of Devices}
The following is a detailed list of the devices that we used to communicate with the Arduino: a 4x3 keypad, a Real-Time Clock unit (RTC), an Adafruit Ultimate GPS Breakout v3, a 16x2 Liquid Crystal Display (LCD), a microSD breakout board, three LSM9DS1s, four force-sensitive resistors (FSRs), and an I2C multiplexer.
The Arduino used a battery pack and 5V power supply to run the devices.\par

\begin{figure}[h]
  \centering
  \includegraphics[scale=0.5]{DAC}
  \caption[DAC]{Arduino data acquisition device}
  \label{fig:x dac}
\end{figure}

\subsection{Arduino and I2C}
The Arduino Mega2560 board combines the Atmel ATmega2560 microcontroller with a variety of ports and interconnect options.
It uses a modified version of the C++ language to run user-defined programs.
In this way, a nearly limitless number of use cases are possible.
For our project, we decided to use the I2C protocol to connect our devices together.\par

I2C, or Inter-Integrated Circuit, is a simple digital communication protocol using two wires for each connection which emphasizes ease-of-use over raw performance.
I2C uses the boss and staff model.
\textit{Boss} refers to the microcontroller in the main Arduino board that sends commands and instructions to the peripheral devices, which are referred to as \textit{staff}.
A competing protocol to I2C, Serial Peripheral Interface, or SPI, has a four wire communication system and is faster than I2C.
However, SPI only allows for one boss capability, whereas I2C has multiple boss capabilities.
Data in SPI can be shared between the boss and staff in two ways---it is full duplex.
Two frequencies are used to transmit and receive information.
I2C, on the contrary, is half duplex, meaning data can flow in both directions, but not simultaneously.
I2C was chosen in our case because it allows for multiple peripheral devices to be connected at one time with minimal wiring while still maintaining a fast enough read speed for each device.\par

One disadvantage of I2C is that it assigns a unique identifier to each device model, meaning it cannot distinguish between multiple of the same device.
To overcome this, we made use of an I2C multiplexer, a device which takes two or more of the same device and assigns them distinct identifiers before connecting them to the boss.\par

\subsection{GPS and RTC}
The GPS receives data from up to 22 satellites and provides data on location, speed, date, and time.
The Real-Time Clock (RTC) was synchronized with the GPS in our data logger to create accurate timing for our data set.
It was also used to get the date and time of when the data were collected for easy synchronization with video data taken from a separate device.\par

\subsection{The LSM9DS1}
The three LSM9DS1 breakout boards can collect gyroscopic, and magnetic field data in addition to acceleration data.
The term gyroscopic data refers to the measurement of angular velocity when rotated around its self defined axes.
Accelerometers cannot directly measure angular displacement, it must be derived which is further explained later in the paper under the section \textbf{\nameref{sec:making}}.
The accelerometers had a range of $\pm \SI{16}{\textit{g}}$ (where $g$ is the acceleration due to gravity), the magnetometer a range of $\pm \SI{12}{\gauss}$ (Gauss), and the gyroscope a range of $\pm \SI[per-mode=fraction]{2000}{deg\per\second}$.
We were able to collect data at up to a rate of $\SI{83}{\Hz}$ for the three devices or about $\SI{1/12}{\milli\second}$ \parencite{adafruit}.

\subsection{Timing and Data Logging}
An SD card was used to store the accelerometer data, the elapsed time obtained from the Arduino \texttt{millis} function, and the real time and date from the RTC.
The \texttt{millis} function gives the number of milliseconds passed since the Arduino program started running, it is a useful and accurate way to sync up the data collected from the different sensors to the time passed since the start of the program.
To monitor the data collection and aid in offline analysis, an LED was installed on each accelerometer and set to blink once at the beginning to signal the start of the program, and continue blinking once every half second thereafter.
This acted as a visual marker to synchronize the data with a video that was being taken of the runner.\par

\begin{figure}[h]
  \centering
  \includegraphics[scale=0.4]{accelerometer_leds}
  \caption[Accelerometer LEDs]{The three accelerometers and their respective LEDs}
  \label{fig:x 3-leds}
\end{figure}


 \begin{figure}[h]
  \centering
  \includegraphics[scale=0.3]{flexiforce_a201}
  \caption[FlexiForce A201]{The FlexiForce A201 force-sensitive resistors used in our study with max 100 lb sensing \parencite{tekscan-sensor}}
  \label{fig:x flexiforce}
\end{figure}

\subsection{Python}
For post processing, Python was chosen as the language to handle the offline analysis of these data.
Python is an interpreted language written by Guido van Rossum.
Part of the philosophy behind Python’s creation was that ``simple is better than complex'' \parencite[Peters, 1999, as cited in][p. 1]{psf}.
Following this mantra, Python has straightforward syntax which makes developing programs much faster than developing those same programs using other languages.
This simplicity often results in programs that take longer to run than in other languages; however, given the nature of offline analysis, this trade-off did not present any issues to this project.
The reduction in time to write code vastly exceeded the time that would have been saved by writing the same code in another language.
In addition to simple syntax, Python has a rich and easy-to-use collection of third-party libraries for data analysis.\par

\section{Making and Using the Parts}
\label{sec:making}

The thought process behind the design to figure out a way to create a system that would be sensitive enough to changes in an angle, but also maintain a degree of rigidity when in a jogging motion.
The part we first focused on was deciding on the orientation of the accelerometers on a person.
As previously mentioned, the angles of interest were those created from the rotation of the vertical lines down the heel and the lower calf with the Achilles tendon as their pivot point.
The other was a change in angle of the surface of the top of the foot with a perpendicular line to the main post of the lower leg.
See \ref{fig:x AOP} for a visual reference.
To capture these angles, accelerometers were placed on the lower calf region, on the heel, and on top of the foot.\par

\begin{figure}[h]
  \centering
  \includegraphics[scale=0.5]{accelerometer_placement}
  \caption[Accelerometer placement]{Accelerometer placement on foot and calf}
  \label{fig:x foot and calf}
\end{figure}

Each accelerometer had a hard plastic back that was 3D printed to secure our individual accelerometers.
These were mainly for the comfort of the runner, but they also served as protection against possible short circuits that might occur on the back of the printed boards' exposed solder joints.\par

\begin{figure}[h]
  \centering
  \includegraphics[scale=0.5, angle=90]{cad_file}
  \caption[CAD file]{Computer-aided design file for the strong accelerometer back}
  \label{fig:x cad file}
\end{figure}

\subsection{Nautical Angles}
\begin{wrapfigure}{R}{0.25\textwidth}
  \begin{center}
  \includegraphics[scale=0.23]{nautical_diagram}
\end{center}
  \caption[Nautical angle diagram]{Nautical angles}
  \label{fig:x nautical}
\end{wrapfigure}

Each accelerometer has its own predefined right handed $x$, $y$, and $z$ coordinate system relative to itself.
The rotational displacement around these axes are referred to as roll, pitch, and yaw.
They are commonly referred to as \textit{nautical angles} or \textit{airplane angles}, as they are used in sailing and aeronautics to describe rotation around a system’s axes.
The following is how roll, pitch, and yaw are related to the axes.\par

A positive roll is a counter-clockwise rotation around the $x$-axis.
A positive pitch is a counter-clockwise rotation around the $y$-axis.
A positive yaw is a counter-clockwise rotation around the $z$-axis.
\hyperref[{fig:x nautical}]{Figure \ref*{fig:x nautical}} illustrates these rotations as we defined them.
\hyperref[{fig:x nautical photos}]{Figure \ref*{fig:x nautical photos}} on the next page shows how the coordinate systems of each accelerometer would be oriented in test setups.\par

\begin{figure}
  \centering
  \subfloat[Yaw illustration with photo]{\label{yaw photo}\includegraphics[scale=0.25]{yaw_photo}} \\
    \subfloat[Pitch illustration with photo]{\label{pitch photo}\includegraphics[scale=0.25]{pitch_photo}} \\
    \subfloat[Roll illustration with photo]{\label{roll photo}\includegraphics[scale=0.25]{roll_photo}}
  \caption[Nautical angle photos]{Illustration of nautical angles with photos}
  \label{fig:x nautical photos}
\end{figure}

\subsection{How Roll and Pitch are Calculated}
As mentioned previously, the accelerometers can only measure an angular velocity.
For an angular displacement to be extracted from from the angular velocity, each angular velocity datum would have to be multiplied by the delta time between each sensor reading.
These resulting angles would then be added to each other sequentially so at any time the sensor would know how much it has tilted.
This method is also known as integration.
While this is the most straightforward method of calculating roll and pitch, it can rarely be used because gyroscopes are prone to drifting and overshoot.
As more and more overshot data are added to the roll and pitch, the readings become less accurate.\par

To alleviate this, another way to calculate roll and pitch is to use the force of gravity that the accelerometers can detect at all times.
This is done by calculating the angle that is created between the coordinate system of the accelerometer and Earth’s coordinate system.
Earth’s downward direction, $z$, is determined by the direction of gravity.
This is what each accelerometer then compares to its $x$ and $y$-axes to calculate roll and pitch.
The idea behind this is visually represented in \hyperref[{fig:x angular displacement}]{Figure \ref*{fig:x angular displacement}}
As a disclaimer, the labels and orientation of the block in \hyperref[{fig:x angular displacement}]{Figure \ref*{fig:x angular displacement}} do not represent coordinate systems and angles exactly.\par

\begin{figure}[h]
  \centering
  \includegraphics[scale=0.5]{angular_displacement}
  \caption[Angular displacement]{Angular displacement between Earth’s and the accelerometers' coordinate systems \parencite{newton}}
  \label{fig:x angular displacement}
\end{figure}

The following equations for roll and pitch were used in the offline analysis:
\begin{align}
  \mathrm{roll} &= \arctan\left( \frac{a_y}{a_z} \right)\label{eq:x roll}\\
  \hfill\nonumber\\
  \mathrm{pitch} &= \arctan\left( \frac{-a_x}{\sqrt{a_y^2+a_z^2}} \right)\label{eq:x pitch}
\end{align}

The $a_x$, $a_y$, and $a_z$ is how the gravitational field vector is separated into components based on the orientation of the accelerometer's coordinate systems relative to it.
A derivation of \ref{eq:x roll} and \ref{eq:x pitch} is beyond the scope of this paper;
for a full treatment, see \cite{pedley}.
Yaw was not used in this study because it is the least accurate method to calculate angular displacement.
It cannot solely use gravity because the gravitational vector points along the $z$-axis of the accelerometer.
When both coordinate systems are aligned, rotating around it provides no change to a reference axis.
The best option would be to use Earth’s magnetic field as a compass and make magnetic north the reference axis.
This would be done with the magnetometer that comes on board the LSM9DS1.
While this works, magnetometers tend to read more slowly and unpredictably than accelerometers, so their readings will be inherently less accurate than those of the other sensors.\par

\subsection{The Complementary Filter}
While the previous method generally yields more consistent results than solely integrating, it can only be used accurately in static and slow moving cases.
This is because if the sensors detect other accelerations that are not from gravity, then those will distort the direction of the gravitational vector to the sensor, which leads to inaccurate readings.
The solution was to combine both methods.
This is what is called a complementary filter, where the angle from integrating acts as a check to the angle being calculated from using Earth’s gravity.
The following equation, taken from \cite{grahn}, is the implementation of the complementary filter:

\begin{equation}
  \mathrm{angle}_{\mathrm{complementary}}=\beta\cdot\mathrm{angle}_{\mathrm{integrate}}+(1-\beta)\cdot\mathrm{angle}_{\mathrm{gravity}}\label{eq:x beta}
\end{equation}

$\beta$ is known as the filter coefficient.
This parameter determines the percent usage of each of the calculation methods.
First, a percent value of the angle by gravity is taken; then, the remaining percentage is taken from the angle by integration.
These two percentages are then added together to get the final, more accurate value.
In our calculations $\beta$ was set to 0.93, or 93\%, because we wanted to use more of the angle from gravity than from integration, as it is more accurate.
The percentage from integration would consequently be 0.07, or 7\%.\par

\subsection{The Kalman Filter}
An even more advanced method utilizes the so-called Kalman filter.
This filter uses Bayesian statistics to better correct for error.
It works in a two-step process.
First the filter predicts the current state variables and their uncertainties using the previous measurement, then, once the current measurement is taken, it updates its estimates using a weighted average.
In this way, the filter creates better and better estimates as it is run.
The traditional Kalman filter assumes the system is linear; however, nonlinear extensions have been developed.\par

The great difficulty in using the Kalman filter is in its setup.
For multi-dimensional systems, such as ours, its use requires the development of several matrices to calculate the state of the system.
Additionally, the algorithm itself is quite tricky to implement.
While several esoteric libraries exist to facilitate its use in Python, not all of them are well-documented, and the ones that are are not necessarily straightforward to use.
Ultimately, we determined that the investment of time required would not be worth it, when the complementary filter gave us good results by itself.\par

\subsection{Force-Sensitive Resistors}
After placing the accelerometers, we chose the placement of the FSRs on the pads of the feet.
Their placement was straightforward as we followed the method outlined in \cite{menz}, (see \hyperref[{fig:x press. map}]{Figure \ref*{fig:x press. map}} from their work).
The placement of the FSRs was also chosen after some subject testing in different shoes.
A foam insert of the running shoe was used to attach the FSRs as shown in \hyperref[{fig:x fsr_placement}]{Figure \ref*{fig:x fsr_placement}}.\par

\begin{figure}[h]
  \centering
  \includegraphics[scale=0.4]{fsr_placement_in_shoe}
  \caption[FSR Placement]{Force-sensitive resistor placement on sole of shoe}
  \label{fig:x fsr_placement}
\end{figure}

The use of only accelerometers to detect PR or SP would be difficult due to factors such as noise and synchronization of the data to the runners gate.
The inclusion of FSRs allowed for better conclusions to be drawn and helped connect the data that were being collected from the accelerometers to the phases of a runner’s gate as seen in \hyperref[{fig:x gait cycle}]{Figure \ref*{fig:x gait cycle}}.
For example, an FSR on the heel would likely get the most force during the heel strike phase.
The FSRs are circular pads which record a change in capacitance when depressed.
This change in capacitance can then be converted to a force.
Each FSR has range from 0 to 100 lbs.\par

\begin{figure}[h]
  \centering
  \includegraphics[scale=0.35]{gait_cycle}
  \caption[The Gait Cycle]{Phases of the human gait cycle \parencite{tekscan-cycle}}
  \label{fig:x gait cycle}
\end{figure}

The way the raw analog to digital conversion (ADC) values were converted to pounds of force was by performing the following calculations and calibration in post processing.
If we know the maximum ADC value, the ADC reference voltage, and the series resistance in the circuit, we can calculate the conductance of the FSR, $G_{FSR}$.

\begin{equation}
  V=\frac{V^{ref}_{ADC}\cdot ADC_{count}}{ADC_{max count}+1}\qquad R_{FSR}=\frac{V^{ref}_{ADC}}{V-1}\cdot R_{series}\qquad G_{FSR}=\frac{1}{FSR}\label{eq:x adc convert}
\end{equation}

Since the relationship between the conductance and force is linear, all we had to do was perform some calibration tests and experimentally find the force for each sensor.
This is also why we chose to use force conductance and not resistance.
We placed multiple objects of known weight on the FSRs and recorded the weight and conductance after converting the ADC count.
Each object was weighed multiple times and for an extended period of time to get the most accurate data.
A linear fit was created for the data and the line model was the relation between the conductance and the force.\par

\begin{figure}[h]
  \centering
  \includegraphics[scale=0.4]{conductance_vs_force}
  \caption[Conductance vs. force]{Graph of conductance vs. force \parencite{tekscan-convert}}
  \label{fig:x cond v f}
\end{figure}

Around 40 data were taken for each of the four masses corresponding to 0.18 lbs, 0.7375 lbs, 1.4125 lbs, and 2 lbs.
You can see in \hyperref[{fig:x best fit}]{Figure \ref*{fig:x best fit}} for the middle 2 masses, we only saw 2 differing conductance values read from the FSRs.
The known mass object that we used for the 2lb mass was a water bottle.
We had to balance this unstable and oddly weighted mass on a small funnel so that the mass would be concentrated on the 0.38 inch diameter FSR.
This is most likely why six different conductance values were measured for the 2 lb mass.
A line of best fit was then used to determine the relationship between conductance and force.\par

\begin{figure}[h]
  \centering
  \includegraphics[scale=0.3]{cond_v_force_best_fit}
  \caption[Conductance vs. force best-fit line]{Line of best fit for experimental data relationship between conductance and force in pounds}
  \label{fig:x best fit}
\end{figure}

We can see in \hyperref[{fig:x best fit}]{Figure \ref*{fig:x best fit}} that the equation found relating conductance in mhos, $G$, to force in pounds, $F$, is $G = 6\times10^7\cdot F - 10^{-7}$.
This equation can be reorganized to solve for force since that is how we will be using it.\par

\begin{equation}
  F=\frac{5}{3}G+\frac{1}{6}\label{eq:x F}
\end{equation}

For data analysis, we used the Matplotlib, NumPy, and Pandas Python libraries.
They allowed for data loading, cleaning, and visualization in only a few lines of code.
Graphs were created for the nautical angles with respect to time to find trends and patterns in someone’s stride, and compared with resting data to look for PR and SR.
A force-map of the FSR data was created to visually show the distribution of forces during a trial.
To gauge the reliability of the sensors, we found the average angles and accelerations from a resting frame using simple calculations.\par

Now with all the data sensors in place, they needed to be connected to a data acquisition device (DAC) to mainly read and record the data to a digital file to be read later by processing software in Python.\par

\section{DAC Casing}
The design of the DAC had a movable lid so we could insert and draw the lid through the grooves on each side of the case.
We also made openings on the lid to make the number pad, LCD, and ribbon cable connections visible, which made the setup of the devices easier and enabled monitoring of the system when collecting data.
There was an opening on the bottom of the case below the battery pack to enable quick removal and reinstallation of the batteries so the system could be reset between trials.\par

\begin{figure}[h]
  \centering
  \includegraphics[scale=0.2]{3d_printed_case}
  \caption[DAC case]{3D-printed case for the data logger}
  \label{fig:x case}
\end{figure}


\section{Field Tests}

Once the data acquisition program was loaded into the PCB, the LCD displayed a message to help verify that the SD card was working properly.
The LED lights then blinked for three seconds to indicate the setup was complete and the program was ready to collect data.
Additionally, an LCD display was programmed to display any error messages when the other devices were initializing.\par

To start data collection, the person being tested pressed the star key (*) on the keypad, and then the data logger started collecting the accelerometer and FSR data.
Once the person has completed running or walking, they will press the pound key (\#) to stop data collection.
Doing this allowed for multiple trials to be taken without extra unneeded data being taken in between trials.\par

Supplemental video was taken to correlate what the data were showing when graphed to what was happening in the real world.
Video recording was performed by having another person follow alongside the runner while the camera was on a rolling cart.
The video, along with the time the DAC program started taking data, was timestamped to synchronize the time scales of the data and the video.\par

Three types of field tests were performed.
The first test had the subject standing still in a neutral position to acquire data on pronation and supination in a static case.
This would also allow for comparison with dynamic cases that followed, providing a baseline against which to compare the data.
The second test had the subject walking, and the third lightly jogging.
The goal of the different tests was to have a variety of data, and to see if pronation or supination would be more obvious in a dynamic or a static case.\par

Since we did not have subjects who had a known medical history of PR or SP in their limbs, we chose to have one of our subjects emulate PR and SP to the best of their ability and then do the three tests again.
These tests will be referred to as their number first, then PR or SP.
