Abnormal foot pronation and supination has been correlated to leg and foot injuries, especially in activities such as running and dance \parencite{willems}.
Our study aims to better understand pronation (PR) and supination (SP) under specific conditions of dynamic motion.
There has been little research and methods developed on studying PR and SP under dynamic motion such as jogging and walking.
PR and SP are classified, in a general sense, as the movement of the subtalar joint, which is the articulation between the talus (ankle bone) and the calcaneum (heel bone) \parencite{griffiths}.
Over angular extension of this junction has been linked to physical ailments such as plantar fasciitis, achilles tendonitis, back pain as well as other issues \parencite{willems}.\par
\begin{figure}
  \centering
  \includegraphics[scale=0.3]{figure_1.jpg}
  \caption[Pronation visualization]{Pronation and supination visualization of the right foot \parencite{difference}}
  \label{fig:x pron. visual}
\end{figure}
