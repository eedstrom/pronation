Measuring the foot pronation of a runner in motion is, however, no trivial task.
To obtain accurate data, we would need to account for a number of factors.
First, we would need to design a reliable and quantitative method for measuring pronation while running.
Second, we would need a way to easily manipulate the data, in order to correct errors, analyze the data in different ways, and elegantly present it to others.
Finally, we would need our method to be repeatable by others, so it would need to use readily available components with easy assembly.
With these considerations in mind, we set out to design our data acquisition unit.\par
PR and SP can be qualitatively described by looking at qualities such as the foot’s arch, a person's shoe wear pattern and surface area contact shape with a flat surface \parencite{erickson}.
The quantification of PR and SP is split between angles that the lower limb creates with itself as well as the ground and the force that is applied by the foot on a sensor.
The angle that is created by the lower leg and a vertical line coming from the ground is described in a short research study from the University of California - Merced.
\ref{fig:x AOP} below from their studies illustrates this angle of interest.\par
\begin{figure}
  \centering
  \includegraphics[scale=0.4]{figure_2.png}
  \caption[Pronation angle]{Pronation and supination classified in humans. Angle of pronation is denoted by AOP \parencite{erickson}}
  \label{fig:x AOP}
\end{figure}
The third angle was created by the shifting of the top of the foot inwards or outwards.
Inwards would indicate PR and outwards would be SP. In a study where they examined associations between foot posture and foot function to foot pain, one of the methods they used to measure PR and SP was with a pressure mat that subjects would stand on \parencite{menz}.
Their results were a heat map that showed a person's high and low points of pressure. Again, here is a visual reference to their work in \ref{fig:x press. map}.\par
\begin{figure}
  \centering
  \includegraphics[scale=0.3]{figure_3.png}
  \caption[Foot pressure map]{Center of pressure calculation from dynamic foot pressure map \parencite{menz}}
  \label{fig:x press. map}
\end{figure}
Based on these static measurements, individuals are often prescribed a shoe intended to correct their stride \parencite{neumann}.
However, recent studies have cast doubt on the reliability of this method \parencite{richards}.
Indeed, Neumann (2017) concludes that “no validated algorithm yet exists to determine the appropriate shoe for an individual that will minimize injury risk while maintaining or improving performance.”
Given the uncertainty over how best to assess an individual’s needs for a particular shoe, and the apparent unreliability of measuring static foot posture, we decided to try developing an alternative method of measuring PR and SP in runners and others.  Our primary goal was to measure foot posture during the act of running itself, seeing as this is the activity during which injury risk is maximized.  This, we hoped, would provide a more realistic measure of a person’s tendency to pronate when running, and may better inform the decision of what type of shoe they should wear.
