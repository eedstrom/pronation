Beginning with \hyperref[{fig:x neutral data}]{Figure \ref*{fig:x neutral data}}, standing data were collected to develop a strong base line for each subject's neutral standing posture.
Subject 1’s standing data show that the accelerometer on the lower calf is recording, on average, a $\SI{-6.8}{\degree}$ angle.
Since a negative rotation corresponds to an angle made to the right of a reference axis and described in \hyperref[{fig:x AOP}]{Figure \ref*{fig:x AOP}}, this would be considered an angle of pronation.
The heel accelerometer makes, on average, a positive 6.8 degree angle with reference axis but since the angle in \hyperref[{fig:x AOP}]{Figure \ref*{fig:x AOP}} of the heel is made by rotating in the opposite direction we are interested in the opposite vertical angle that the heel accelerometer makes.
Since $\SI{6.8}{\degree}$ is made by the heel accelerometer to the right of the reference, Subject 1 is pronating.
Since both angles do not not exceed $\SI{9}{\degree}$, Subject 1’s pronation does not go past safe normal values.
While we established this baseline, it is not fully indicative of pronation or supination during motion like walking or running.
The static cases were compared to the dynamic motion cases that followed.\par

\hyperref[{fig:x neutral data}]{Figure \ref*{fig:x neutral data}} was of Subject 1 walking slowly across the floor over the course of 30 seconds.
The peaks of FSR 3 align with the yaw angle of accelerometer 2’s peaks.
The peaks of the FSR data correspond to when that sensor is being depressed by the foot.
FSR 3 is placed in the toe box, meaning at every peak that is detected, the subject would be at the end contact phase of their stride.
This is mainly important because it helps identify frames of contact in the data, which is important when testing for PR and SP.
At these peaks for the accelerometer 2, the average angular displacement is $\SI{50}{\degree}$.
If we subtract that from our $\SI{90}{\degree}$ offset, we would get a change in angle of $\SI{40}{\degree}$ in the negative direction.
We cannot assume that Subject 1 pronates $\SI{40}{\degree}$ just from this raw analysis alone, however.
There was a significant amount of experimental error from the moving motion of the foot and the inconsistency of our magnetic sensor readings.\par

This inconsistency in the magnetic sensors becomes obvious when looking at \hyperref[{fig:x slow frame}]{Figure \ref*{fig:x slow frame}}.
The yaw data for both accelerometers jump rapidly between their maximum and minimum values, much faster than any human could.
The yaw data must be calculated with Earth's magnetic field because gravity points in the same direction whether upside down or right side up.
The issue also has to do with the type of motion; in this case, jogging.
This can be interpreted to say that any type of motion will result in a poor reading from the magnetometer.
Another contributing factor could have been the limitations of the hardware.
We were not able to read out the magnetometer as quickly as the other sensors, so we had to propagate the previous magnetic measurements to the new accelerometers reading until the magnetometer was ready again.
These placeholder values could have caused outlier data to be propagated, skewing the calculation of yaw.\par

From the filtered data, the peaks in the plots are much more consistent.
However, they do not signify much without a reference.
The original idea to determine whether one was pronating or supinating was to compare one’s roll, pitch, and yaw with those of one who had already been determined to be pronating or supinating.
To attempt this, individuals intentionally pronated or supinated while standing, and while in motion.
These angles could then be used for comparison, specifically at key moments in one’s stride.
These moments include when one’s foot leaves the ground and when it touches back down.
While this method could be utilized, it has several difficulties that prevented its use in this project.
First, defining pronation and supination angles by one intentionally pronating and supinating is inconsistent without a more definitive, objective measure.
The objective of this device is to determine this angle from a specific measurement; however, defining this measurement using the device carries uncertainties with it.
With more fine-tuned control of the device, along with instant access to the values, this is possible, and it is a path future researchers could benefit from exploring further.\par

Instead, the method chosen was to assign values from previous studies on this topic.
The pronating and supinating angles using those values and the positioning of the accelerometers, were determined.
The values defined in \cite{genova}, became the cutoff values.
An example of how these cutoff values were utilized is shown below.\par

\begin{figure}[p]
  \centering
  \includegraphics[scale=0.4]{region_jog}
  \caption[Pronation and supination data]{Pronation and supination during motion}
  \label{fig:x pron. motion}
\end{figure}

The first plot shown in \hyperref[{fig:x pron. motion}]{Figure \ref*{fig:x pron. motion}} displays the times during which the subject is pronating for the jogging data.
The green line represents the pronation cutoff determined by Genova and Gross.
Any point above this line is defined as a point in which the subject is pronating.
The second plot displays the same information for supinating; however, the time when the subject supinates is defined as the points below the green cutoff.

\begin{itemize}
  \item The standing plot suggests pronation. The angles subtract to around $\SI{14}{\degree}$.
    \begin{itemize}
      \item The accelerometers were not installed on the person parallel to the ground in the standing phase.
    \end{itemize}
  \item Foot to ground contact time compared to foot in air time.
    \begin{itemize}
      \item The peaks do not correspond to when her foot is on the ground
      \item Separate the on-the-ground pronation from the non-impactful off-the-ground pronation.
      \item On the ground data similar to rest data. Approximately $\SI{14}{\degree}$.
      \item The reason for spike values while in air is because it is the greatest time of lateral acceleration.
	\begin{itemize}
	  \item Even with the complementary filter, too much lateral acceleration could greatly increase the $g$ vector into the $x$ or $y$ directions.
	\end{itemize}
    \end{itemize}
  \item Running graph is similar to walking graph.
    \begin{itemize}
      \item But there does seem to be an increase in pronation from $\SI{14}{\degree}$ to $\SI{20}{\degree}$ on average.
      \item For the running trial, the extreme data could have resulted from the refresh rate of the accelerometers being too slow.
    \end{itemize}
\end{itemize}

